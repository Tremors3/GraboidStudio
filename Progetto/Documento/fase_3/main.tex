\documentclass{article}
\usepackage[italian]{babel}
\usepackage[utf8x]{inputenc}
\usepackage[T1]{fontenc}
\usepackage{graphicx}
\usepackage[colorinlistoftodos]{todonotes}
\usepackage[colorlinks=true, allcolors=tudelftblue]{hyperref} %sets hyperlink colour
\usepackage{caption}
\usepackage{subcaption}
\usepackage{xcolor}
\usepackage{roboto} % for Roboto Slab font
\usepackage{float}
\usepackage{titling} 
\usepackage{blindtext}\usepackage{titlesec}
\usepackage[square,sort,comma,numbers]{natbib}
\usepackage[colorinlistoftodos]{todonotes}
\usepackage{tikz}
\usepackage{geometry}
\usepackage{sectsty}
\usepackage{amsmath}
\usepackage{tikzpagenodes}
\usepackage{booktabs}
\usepackage{listings}
\usepackage{makecell} 
\usepackage{multirow}
\usepackage{longtable}
\usepackage{svg}
\usepackage{enumitem}
\definecolor{tudelftdarkblue}{RGB}{0,0,0}
\definecolor{tudelftcyan}{RGB}{209,65,36}
\definecolor{tudelftblue}{RGB}{99, 102, 106}
\geometry{a4paper, margin=2cm}
\allsectionsfont{\color{black}} %sets colour for all headers
\usepackage{helvet}
\renewcommand{\familydefault}{\sfdefault}
\sectionfont{\fontfamily{RobotoSlab-TLF}\selectfont}

\newcounter{countertabelle}
\setcounter{countertabelle}{0}
\newcommand{\ctabelle}{\addtocounter{countertabelle}{1}\thecountertabelle}

\newcounter{counteroperazioni}
\setcounter{counteroperazioni}{0}
\newcommand{\coperazioni}{\addtocounter{counteroperazioni}{1}\thecounteroperazioni}
%%%%%%%%%%%%%%%%%%%%%%%%%%%%%%%%%%%%%%%%%%%%%%%%%%%%%%%%%
\begin{document}

\begin{titlepage}
    \fontfamily{RobotoSlab-TLF}\selectfont 
%%%%%%%%%%%%%%%%%%%%%%%%%%%%%%%%%%%%%%%%%%%%%%%%%%%%%%%%%%UNCOMMENT THE FOLLOWING FOR LESS "PLAIN" TITLE PAGE (SELECT WITH MOUSE AND PRESS CTRL AND /)

    % \begin{tikzpicture}[remember picture,overlay]
    %     % Set seed for random number generator
    %     \pgfmathsetseed{4}
    %     % Define the text area to avoid
    %     \path (current page text area.south west) rectangle (current page text area.north east);
    %     % Adding circles spread over the entire page
    %     \foreach \x in {1,...,1000}
    %         \draw[tudelftdarkblue] (current page.south west) ++(rand*\paperwidth,rand*\paperheight) circle (rand*0.3);
    %     % Define coordinates for the corners of the white box
    %     \coordinate (A) at ([shift={(-8cm,12cm)}]current page.center);
    %     \coordinate (B) at ([shift={(8cm,-5cm)}]current page.center);
    %     % Draw the white background box
    %     \fill[white] (A) rectangle (B);
    %     % Adding equations as background features
    %     \node[anchor=center,rotate=20,text=tudelftcyan] at ([shift={(-7cm,-2cm)}]current page.center) {\fontsize{18}{22}\selectfont
    %     $\nabla^2 T - \frac{1}{\alpha}\frac{\partial T}{\partial t} = 0$};
    %     \node[anchor=center,rotate=-15,text=tudelftcyan] at ([shift={(5cm,-4cm)}]current page.center) {\fontsize{18}{22}\selectfont
    %     $\frac{\partial \rho}{\partial t} + \nabla \cdot (\rho \mathbf{v}) = 0$};
    %     \node[anchor=center,rotate=20,text=tudelftcyan] at ([shift={(6cm,4cm)}]current page.center) {\fontsize{18}{22}\selectfont
    %     $a^2 + b^2 = c^2$};
    %     \node[anchor=center,rotate=10,text=tudelftcyan] at ([shift={(7cm,-2cm)}]current page.center) {\fontsize{18}{22}\selectfont
    %     $E = \frac{\sigma}{\varepsilon}$};
    %     \node[anchor=center,rotate=-10,text=tudelftcyan] at ([shift={(-6cm,4cm)}]current page.center) {\fontsize{18}{22}\selectfont
    %     $F = ma$};
    %     \node[anchor=center,rotate=5,text=tudelftcyan] at ([shift={(-4cm,-5cm)}]current page.center) {\fontsize{18}{22}\selectfont
    %     $Q = -\frac{kA}{\mu} \frac{\Delta P}{L}$};
    % \end{tikzpicture}
%%%%%%%%%%%%%%%%%%%%%%%%%%%%%%%%%%%%%%%%%%%%%%%%%%%%%%%%%%
    \vspace*{3cm}
    
    \centering
    {\Huge \textbf{\textcolor{black}{Graboid Studio}}}\\[1.5cm]
    \textsc{Nome Corso}\\[0.5cm]
    \text{\large Basi di Dati}\\[2cm]
    
    {\Large \textbf{\textcolor{tudelftdarkblue}{Autori}}}\\[0.5cm]
    \begin{tabular}{c}
        \Large \textcolor{tudelftdarkblue}{Menozzi Matteo (176906)} \\
        \Large \textcolor{tudelftdarkblue}{Patrini Andrea (176907)} \\
        \Large \textcolor{tudelftdarkblue}{Turci Sologni Enrico (176187)} \\
    \end{tabular}\\[2cm]
    
    {\Large \textcolor{tudelftdarkblue}{\today}}
    
    \vfill
    \begin{center}
        \includegraphics[width=0.6\textwidth]{images/Logo_C_Positivo_Colore.png}
    \end{center}
\end{titlepage}
\tableofcontents

\newpage
\section{Analisi dei requisiti e progetto delle viste}
Questo progetto si occupa di descrivere il funzionamento di uno studio di registrazione e delle sue principali attività. Si occuperà dell'acquisizione e rielaborazione di registrazioni audio per conto di artisti (Musicisti Solisti, Cantanti Solisti, Gruppo).  Le migliori registrazioni sono selezionate ed inserite all’interno di collezioni (Singoli, EP, LP o Album) e ridistribuite in formato digitale. \\ \\ Lo studio di registrazione si trova in un edificio dove sono locate diverse sale di registrazione, le quali dispongono delle appropriate attrezzature per consentire lo svolgimento delle normali attività. Tecnici specializzati si occupano di registrare e manipolare i brani. Per gestire lo scheduling delle registrazioni e per evitare sovrapposizioni delle ore di registrazione, vi è un sistema di prenotazioni e un apposito operatore in grado di gestire tali prenotazioni. \\ \\
Nel seguito considereremo perciò le seguenti diverse classi di utenza:
\begin{itemize}
    \item \textbf{Artista}.
    \item \textbf{Tecnico}.
    \item \textbf{Operatore}.
\end{itemize}

%%%%%%%%%%%%%%%%%%%%%%%%%%%%%%%%%%%%%%%%%%%%%%%%%%%%%%%%%
\newpage
\subsection{Vista Artista}
\subsubsection{Formulazione e analisi dei requisiti per gli Artisti}

\renewcommand*{\arraystretch}{1.4}
\begin{longtable}{|r|p{.8\linewidth}|}
  \hline
  & \multicolumn{1}{c|}{\textbf{Requisiti richiesti dagli Artisti}}
  \endhead
  \hline
  \multirow{1}{1em}{\raggedleft \\[-8.5pt] 1\\2\\3\\4\\5\\6\\7\\8\\9\\10\\11\\12\\13\\14\\15\\16\\17\\18\\19\\20\\21\\22\\23\\24\\25\\26\\27} 
  & L'Artista è l'ente che contatta lo studio per registrare professionalmente le proprie canzoni. Ogni Artista deve essere un Solista (cantante o musicista) oppure un Gruppo di Solisti. Ogni artista, di entrambi i tipi, è caratterizzato da un nickname, una data di registrazione, note testuali sull'artista, molteplici numeri di cellulare, molteplici email, iban. Nel dettaglio i Solisti saranno ulteriormente caratterizzati da un nome, un cognome, una data di nascita; il gruppo necessita soltanto della data di formazione. Deve essere possibile tenere traccia della data nella quale il Solista entra a far parte del gruppo, un solista può far parte di un solo gruppo alla volta. Un'altra figura di notevole importanza è il Produttore. Il Produttore è un Solista che nel corso della produzione di un disco ricopre diversi ruoli, tra i quali figurano la supervisione delle sessioni in studio di registrazione, la preparazione e la guida dei musicisti e la supervisione dei processi di mixaggio e mastering. E' possibile che un Solista, che non faccia parte del gruppo che ha pubblicato una canzone, partecipi comunque nella sua composizione. Deve quindi esserne riportata la sua partecipazione. Non è possibile che un Gruppo partecipi alla composizione di una canzone appartenente ad un altro artista. La canzone è un pezzo musicale più o meno esteso caratterizzata da un codice-univoco, da un titolo, dalla lunghezza in secondi, dalla data di registrazione, e dal codice della collezione a cui appartiene. Ogni canzone può appartenere ad una sola collezione di canzoni. La collezione è una raccolta di canzoni. Ogni collezione è contraddistinta dal codice-univoco dell'artista che la possiede, dalla data di inizio e termine delle registrazioni, da un titolo, dalla tipologia e dallo stato. La collezione può essere in stato di produzione (le canzoni stanno ancora venendo registrate) ed in stato di Pubblicazione (tutte le canzoni sono state registrate). Una collezione una volta pubblicata diventa immutabile, cioè non è più possibile aggiungere ad essa nuove canzoni. La collezione può far parte di una delle tre seguenti tipologie: il Singolo che comprende da una a tre canzoni, l'EP (o Extended Play) che comprende dalle 4 alle 5 canzoni con una durata complessiva minore di mezz'ora e infine l'Album (o LP cioè Long Play) che contiene un numero di canzoni maggiore dell'EP e di durata complessiva tra la mezz'ora e l'ora.
 \\
  \hline
\end{longtable}

\textbf{Tabella 1.\ctabelle} Specifiche in linguaggio naturale per gli Artisti

\subsubsection{Dizionario dei concetti per gli Artisti} 

\renewcommand*{\arraystretch}{1.4}
\begin{longtable}{|p{.15\linewidth}|p{.25\linewidth}|p{.2\linewidth}|p{.2\linewidth}|}
    \hline
    \textbf{Termine} & \textbf{Descrizione} & \textbf{Sinonimi} & \textbf{Collegamenti}
    \endhead
    \hline
    Artista & Ente che contatta lo studio per registrare professionalmente le proprie canzoni & Cliente & Collezione, Prenotazione \\ \hline
    Solista & Artista singolo cantante o musicista & Musicista, Cantante & Gruppo\\ \hline
    Gruppo & Artista composto da un insieme di musicisti & Band & Solista \\ \hline
    Produttore & Solista che supervisiona le sessioni di registrazione dei musicisti e dei processi di mixaggio e mastering & & Collezione \\ \hline
    Collezione & Una raccolta di canzoni; il prodotto finale dell'Artista & Produzione, Album & Artista, Produttore, Canzone \\ \hline
    Canzone & Pezzo musicale più o meno esteso & Brano, Traccia & Collezione  \\ \hline
    Prenotazione & Impegno di un Cliente ad occupare uno studio. & Impegno & Artista \\ \hline
\end{longtable}
\textbf{Tabella 1.\ctabelle} Dizionario dei concetti

\subsubsection{Operazioni per gli Artisti}
\setcounter{counteroperazioni}{0}
\begin{itemize}[labelindent=1.5em,labelsep=.5cm,leftmargin=*]
    \item [\textbf{A\coperazioni)}] ELENCARE LE COMPOSIZIONI A CUI HA PARTECIPATO UN ARTISTA \\ Vengono elencate tutte le canzoni a cui ha partecipato un artista.
    \item [\textbf{A\coperazioni)}] ELENCARE GLI ARTISTI CHE HANNO PARTECIPATO ALLA CREAZIONE DI UNA CANZONE \\ Data una certa canzone vengono mostrate le informazioni degli artisti che hanno partecipato ad essa.

    \item [\textbf{A\coperazioni)}] CREARE UNA PRENOTAZIONE \\ Un artista effettua una prenotazione specificando: i giorni, le fasce orarie scelte e le relative sale. 
    \item [\textbf{A\coperazioni)}] ANNULLARE UN ORDINE \\ Un artista può annullare un ordine specifica fino a sette giorni prima della data dell'ordine.
    \item [\textbf{A\coperazioni)}] VISUALIZZARE LE INFORMAZIONI RELATIVE AL PAGAMENTO DI UN ORDINE \\ Vengono visualizzate le informazioni nome, cognome del cliente, data in cui è stata effettuata l'ordine, lo stato "pagato", "da pagare", costo totale dell'ordine. 
    \item [\textbf{A\coperazioni)}] ELENCARE LE SALE DISPONIBILI \\ Viene visualizzato un elenco di tutte le sale libere per una certa data/ora.
    % \item [\textbf{A\coperazioni)}] TITOLO \\
\end{itemize}



\subsubsection{Progetto schema E/R per gli Artisti}
\begin{center}
    \includesvg[width=.7\textwidth]{images/schema_scheletro_artista.svg}
\end{center}
\textbf{Raffinamenti top-down}:
\begin{itemize}
    \item 
\end{itemize}
\textbf{Raffinamenti bottom-up}:
\begin{itemize}
    \item 
\end{itemize}
\textbf{Raffinamenti inside-out}:
\begin{itemize}
    \item 
\end{itemize}

\renewcommand*{\arraystretch}{1.4}
\begin{longtable}{|p{.15\linewidth}|p{.25\linewidth}|p{.25\linewidth}|p{.15\linewidth}|}
    \hline
    \textbf{Entità} & \textbf{Descrizione} & \textbf{Attributi} & \textbf{Identificatore} 
    \endhead 
    \hline
    Operatore & Operatore che gestisce gli ordini & Codice Fiscale, Nome, Cognome, Data nascita, Data assunzione, numero telefono, iban & Codice Fiscale \\ \hline
    Ordine & Ordine per una sala effettuata da un cliente (artista) & Codice, Numero giorni di prenotazione rimanenti, Data  prenotazione & Codice  \\ \hline
    Pagamento & Pagamento per un ordine effettuata/in attesa di un artista & Codice, Stato, Metodo, Costo totale & Codice\\ \hline
    Tipologia & Specifica la tipologia di prenotazione: oraria, giornaliera, settimanale o mensile & Codice, Valore, N giorni da prenotare & Codice \\ \hline
    Prenotazione & Prenotazione compone l'ordine con il lasso di tempo e la sala & Codice, Giorno, Annullata & Codice \\ \hline
    Prenotazione Giornaliera & Prenotazione Giornaliera è una generalizzazione di prenotazione & & \\ \hline
    Prenotazione Oraria & Prenotazione Oraria è una generalizzazione di prenotazione & & \\ \hline
    Artista & L'Artista effettua la prenotazione & Codice, Tipo & Codice \\ \hline
    Sala & Sala di registrazione prenotata da un artista & Codice & Codice \\ \hline
    Oraria & Intervallo di tempo per il tipo di prenotazione oraria & Ora inizio, Ora fine & Ora inizio + Codice (Prenotazione Oraria) \\ \hline 
\end{longtable}

\renewcommand*{\arraystretch}{1.4}
\begin{longtable}{|p{.15\linewidth}|p{.25\linewidth}|p{.2\linewidth}|p{.2\linewidth}|}
    \hline
    \textbf{Relazione} & \textbf{Descrizione} & \textbf{Entità coinvolte} & \textbf{Attributi} 
    \endhead 
    \hline
     Associato & Associa una prenotazione a un operatore & Operatore, Prenotazione &  \\ \hline
     Gestisce & Associa un pagamento a una prenotazione & Pagamento, Prenotazione &  \\ \hline
     Appartenenza & Associa una prenotazione a una tipologia & Tipologia, Prenotazione &  \\ \hline
     Effettua & Associa un artista a una prenotazione & Artista, Prenotazione &  \\ \hline
     Composizione & Associa una riservazione a una prenotazione & Riservazione, Prenotazione &  \\ \hline
     Composizione & Associa una riservazione oraria alla sua fascia oraria & Riservazione oraria, Oraria &  \\ \hline
     Oggetto di & Associa una riservazione giornaliera a una sala & Riservazione giornaliera, Sala &  \\ \hline
     Oggetto di & Associa una fascia oraria a una sala & Oraria, sala &  \\ \hline
\end{longtable}

%%%%%%%%%%%%%%%%%%%%%%%%%%%%%%%%%%%%%%%%%%%%%%%%%%%%%%%%%
\newpage
\subsection{Vista Tecnico}
\subsubsection{Formulazione e analisi dei requisiti per i Tecnici}

\renewcommand*{\arraystretch}{1.4}
\begin{longtable}{|r|p{.8\linewidth}|}
  \hline
  & \multicolumn{1}{c|}{\textbf{Requisiti richiesti dai Tecnici}}
  \endhead
  \hline
  \multirow{1}{1em}{\raggedleft \\[-8.5pt] 1\\2\\3\\4\\5\\6\\7\\8\\9\\10\\11\\12\\13} 
  & I Tecnici sono coloro che lavorano negli studi e si occupano di acquisire e modificare le canzoni registrate. Sono caratterizzati da un nome, un cognome, la data di nascita, la data di assunzione, molteplici numeri di telefono, molteplici email, un iban. I Tecnici devono specializzarsi in Tecnico del Suono oppure in Fonico; le due figure possono anche sovrapporsi. Il fonico è il tecnico che si occupa di mixaggi, livellazione del suono e aggiunta di effetti post registrazione; mentre il Tecnico del Suono cerca di ottenere un audio della migliore qualità possibile. Entrambe le due entità non caratterizzano ulteriormente il tecnico. Ciascun tecnico viene assegnato ad uno e uno solo studio. Uno studio di registrazione consiste in due spazi, una Sala di Controllo; e una Sala di Registrazione insonorizzata. La Sala di Registrazione disporrà delle attrezzature elettroniche adeguate per registrare il suono; analogamente la Sala di Controllo disporrà delle appropriate attrezzature per editare e immagazzinare le canzoni registrate. Ciascuno studio può avere dagli 1 ai 2 tecnici.
 \\
  \hline
\end{longtable}
\textbf{Tabella 1.\ctabelle} Specifiche in linguaggio naturale per i Tecnici

\subsubsection{Dizionario dei concetti per i Tecnici}
\renewcommand*{\arraystretch}{1.4}
\begin{longtable}{|p{.15\linewidth}|p{.25\linewidth}|p{.2\linewidth}|p{.2\linewidth}|}
    \hline
    \textbf{Termine} & \textbf{Descrizione} & \textbf{Sinonimi} & \textbf{Collegamenti}
    \endhead
    \hline
    Tecnico & Il tecnico lavora negli studi e si occupa di acquisire e modificare le canzoni registrate  & Fonico, Tecnico del Suono & Studio, Produzione \\ \hline
    Studio & Uno studio consiste in due spazi: una Sala di Controllo e una Sala di Registrazione & Sala & Tecnico \\ \hline
    Sala di controllo & Sala dove i tecnici possono editare e mixare le registrazioni & Regia &  \\ \hline
    Sala di registrazione & Sala specializzata nella registrazione del suono & Live room &  \\ \hline
    Produzione & Una raccolta di canzoni; il prodotto finale dell’Artista & Collezione, Album & Tecnico\\ \hline
\end{longtable}
\textbf{Tabella 1.\ctabelle} Dizionario dei concetti

\setcounter{counteroperazioni}{0}
\subsubsection{Operazioni per i Tecnici}
\begin{itemize}[labelindent=1.5em,labelsep=.5cm,leftmargin=*]
    \item [\textbf{T\coperazioni)}] ELENCARE LE PRODUZIONI A CUI HA PARTECIPATO UN TECNICO \\ Viene visualizzato un elenco di produzioni a cui ha partecipato un tecnico contenente: le produzioni, artisti, produttori, tecnici, titolo, data di creazione, lo stato della collezione ("produzione" o "pubblicazione"), la tipologia ("Singolo", "EP", "Album").
    \item [\textbf{T\coperazioni)}] ELENCARE LE ATTREZZATURE DI UNA SALA \\ Vengono elencate le attrezzature presenti in una data sala.
    \item [\textbf{T\coperazioni)}] CREAZIONE DI UNA PRODUZIONE \\ Inserimento del titolo, viene indicato l'artista che la possiede e viene impostato lo stato di "produzione".
    \item [\textbf{T\coperazioni)}] PUBBLICAZIONE DI UNA PRODUZIONE \\ Viene impostata per una data produzione lo stato di "pubblicazione" che va ad indicare l'immutabilità di essa e viene impostata la data di termine delle registrazioni. 
    \item [\textbf{T\coperazioni)}] INSERIRE I DATI DI UNA CANZONE\\ Inserimento informazioni generali della canzone quali artisti, produttori che hanno partecipato alla creazione della canzone, titolo, durata in secondi e data di registrazione.
    \item [\textbf{T\coperazioni)}] INSERIRE UNA CANZONE IN UNA PRODUZIONE \\ Una volta terminata la "stesura" di una canzone essa deve essere inserita nella "collezione" di cui fa parte.
    % \item [\textbf{T\coperazioni)}] TITOLO \\
\end{itemize}

\subsubsection{Progetto schema E/R per i Tecnici}
\begin{center}
    \includesvg[width=.8\textwidth]{images/schema_scheletro_tecnico.svg}
\end{center}
\textbf{Raffinamenti top-down}:
\begin{itemize}
    \item 
\end{itemize}
\textbf{Raffinamenti bottom-up}:
\begin{itemize}
    \item 
\end{itemize}
\textbf{Raffinamenti inside-out}:
\begin{itemize}
    \item 
\end{itemize}
%%%%%%%%%%%%%%%%%%%%%%%%%%%%%%%%%%%%%%%%%%%%%%%%%%%%%%%%%

\newpage
\subsection{Vista Operatore}
\subsubsection{Formulazione e analisi dei requisiti per gli Operatori}

\renewcommand*{\arraystretch}{1.4}
\begin{longtable}{|r|p{.8\linewidth}|}
  \hline
  & \multicolumn{1}{c|}{\textbf{Requisiti richiesti dagli Operatori}}
  \endhead
  \hline
  \multirow{1}{1em}{\raggedleft \\ [-8.5pt] 1\\2\\3\\4\\5\\6\\7\\8\\9\\10\\11\\12\\13\\14\\15\\16\\17\\18\\19\\20\\21} 
  & L'operatore è il dipendente che si occupa di rispondere alle chiamate dei clienti. Lavora presso l'edificio dello studio in un locale apposito. Egli è caratterizzato da un nome, un cognome, una data di nascita, una data di assunzione, molteplici numeri di telefono, molteplici email, un iban. Inoltre tra le sue competenze troviamo la registrazione di nuovi ordini inoltrabili tramite chiamata telefonica, la loro supervisione ed eventuale cancellazione. L'operatore deve supervisionare anche lo stato di pagamento degli ordini. Oggetto degli ordini è una sala di registrazione dello studio. Lo studio si impegna a riservare la sala al cliente che ha effettuato l'ordine. Più nel dettaglio lo studio non registra nuovi ordini che impegnano una sala se è già riservata ad altri clienti nella stessa fascia temporale. Lo studio di registrazione non può essere prenotato per meno di un'ora. E' possibile prenotare al massimo una settimana prima del giorno di cui si vuole fare la prenotazione. Il cliente è obbligato a pagare entro tre giorni dopo aver effettuato l'ordine, in caso contrario esso viene annullato. Il cliente è costretto a pagare l'intera cifra dell'ordine in un'unica rata. Un ordine è caratterizzato dal cliente, dalla data di effettuazione dell'ordine, dallo stato di pagamento e di cancellazione e dalla tipologia scelta tra le seguenti: oraria (pagata ad ore), caratterizzata dal giorno e dall'ora di inizio e fine del periodo di impegno; giornaliera, avente prezzo fisso, che al contrario della prima impegna la stanza per l'intera giornata; settimanale, avente prezzo fisso, che mette a disposizione del cliente 7 giorni lavorativi a sua scelta (piano settimanale); mensile, avente prezzo fisso, che mette a disposizione del cliente 30 giorni lavorativi a sua scelta (piano mensile). Non è necessario utilizzare i giorni tutti in fila ma possono essere sfruttati nell'arco di 90 giorni.
 \\
  \hline
\end{longtable}
\textbf{Tabella 1.\ctabelle} Specifiche in linguaggio naturale per gli Operatori

\subsubsection{Dizionario dei concetti per gli Operatori}
\renewcommand*{\arraystretch}{1.4}
\begin{longtable}{|p{.15\linewidth}|p{.25\linewidth}|p{.2\linewidth}|p{.2\linewidth}|}
    \hline
    \textbf{Termine} & \textbf{Descrizione} & \textbf{Sinonimi} & \textbf{Collegamenti} \endhead \hline
    Operatore & Dipendente che si occupa di rispondere alle chiamate dei clienti &  &  Prenotazione\\ \hline
    Cliente & Artista che richiede una prenotazione & Artista & Prenotazione \\ \hline
    Prenotazione & Impegno di un Cliente ad occupare uno studio.  & Impegno & Studio, Pagamento, Operatore \\ \hline
    Pagamento & Mantiene le informazioni relative allo stato del pagamento. & Versamento & Prenotazione   \\ \hline
    Studio & Uno studio consiste in due spazi: una Sala di Controllo e una Sala di Registrazione & Sala& Prenotazioni \\ \hline
\end{longtable}
\textbf{Tabella 1.\ctabelle} Dizionario dei concetti

\setcounter{counteroperazioni}{0}
\subsubsection{Operazioni per gli Operatori}
\begin{itemize}[labelindent=1.5em,labelsep=.5cm,leftmargin=*]
    \item [\textbf{O\coperazioni)}] CREARE UN ORDINE \\ L'operatore registra gli ordini previo accordo con il cliente tramite chiamata telefonica.
    \item [\textbf{O\coperazioni)}] ANNULLARE UN ORDINE \\ L'operatore ha il diritto di annullare un ordine se il cliente non paga entro tre giorni dopo aver effettuato l'ordine previa chiamata telefonica.
    \item [\textbf{O\coperazioni)}] CREARE UNA PRENOTAZIONE \\ L'operatore registra le prenotazioni previo accordo con il cliente tramite chiamata telefonica.
    \item [\textbf{O\coperazioni)}] ANNULLARE UNA PRENOTAZIONE \\ L'operatore può annullare una prenotazione previo accordo con il cliente tramite chiamata telefonica.
    \item [\textbf{O\coperazioni)}] VISUALIZZARE LE INFORMAZIONI RELATIVE AL PAGAMENTO DI UN ORDINE \\ Vengono visualizzate le informazioni nome, cognome del cliente, data in cui è stata effettuato l'ordine, lo stato "pagato", "da pagare", costo totale dell'ordine. 
    \item [\textbf{O\coperazioni)}] ELENCARE GLI ORDINI CHE NON SONO ANCORA STATI PAGATI \\ Viene visualizzato un elenco di ordini non pagati e informazioni di chi ha fatto l'ordine: nome, cognome, telefono, data di effettuazione dell'ordine.
    \item [\textbf{O\coperazioni)}] ELENCARE LE SALE DISPONIBILI \\ Viene visualizzato un elenco di tutte le sale libere per una certa data/ora.
    % \item [\textbf{O\coperazioni)}] TITOLO \\
\end{itemize}

\subsection{Progetto schema E/R per gli Operatori}
\begin{center}
    \includesvg[width=1\textwidth]{images/schema_scheletro_operatore.svg}
\end{center}
\textbf{Raffinamenti top-down}:
\begin{itemize}
    \item 
\end{itemize}
\textbf{Raffinamenti bottom-up}:
\begin{itemize}
    \item 
\end{itemize}
\textbf{Raffinamenti inside-out}:
\begin{itemize}
    \item 
\end{itemize}

\renewcommand*{\arraystretch}{1.4}
\begin{longtable}{|p{.15\linewidth}|p{.25\linewidth}|p{.25\linewidth}|p{.15\linewidth}|}
    \hline
    \textbf{Entità} & \textbf{Descrizione} & \textbf{Attributi} & \textbf{Identificatore} 
    \endhead 
    \hline
    Operatore & Operatore che gestisce gli ordini & Codice Fiscale, Nome, Cognome, Data nascita, Data assunzione, numero telefono, iban & Codice Fiscale \\ \hline
    Ordine & Ordine per una sala effettuata da un cliente (artista) & Codice, Numero giorni di prenotazione rimanenti, Data  prenotazione & Codice  \\ \hline
    Pagamento & Pagamento per un ordine effettuata/in attesa di un artista & Codice, Stato, Metodo, Costo totale & Codice\\ \hline
    Tipologia & Specifica la tipologia di prenotazione: oraria, giornaliera, settimanale o mensile & Codice, Valore, N giorni da prenotare & Codice \\ \hline
    Prenotazione & Prenotazione compone l'ordine con il lasso di tempo e la sala & Codice, Giorno, Annullata & Codice \\ \hline
    Prenotazione Giornaliera & Prenotazione Giornaliera è una generalizzazione di prenotazione & & \\ \hline
    Prenotazione Oraria & Prenotazione Oraria è una generalizzazione di prenotazione & & \\ \hline
    Artista & L'Artista effettua la prenotazione & Codice, Tipo & Codice \\ \hline
    Sala & Sala di registrazione prenotata da un artista & Codice & Codice \\ \hline
    Oraria & Intervallo di tempo per il tipo di prenotazione oraria & Ora inizio, Ora fine & Ora inizio + Codice (Prenotazione Oraria) \\ \hline 
\end{longtable}

\renewcommand*{\arraystretch}{1.4}
\begin{longtable}{|p{.15\linewidth}|p{.25\linewidth}|p{.2\linewidth}|p{.2\linewidth}|}
    \hline
    \textbf{Relazione} & \textbf{Descrizione} & \textbf{Entità coinvolte} & \textbf{Attributi} 
    \endhead 
    \hline
     Associato & Associa una prenotazione a un operatore & Operatore, Prenotazione &  \\ \hline
     Gestisce & Associa un pagamento a una prenotazione & Pagamento, Prenotazione &  \\ \hline
     Appartenenza & Associa una prenotazione a una tipologia & Tipologia, Prenotazione &  \\ \hline
     Effettua & Associa un artista a una prenotazione & Artista, Prenotazione &  \\ \hline
     Composizione & Associa una riservazione a una prenotazione & Riservazione, Prenotazione &  \\ \hline
     Composizione & Associa una riservazione oraria alla sua fascia oraria & Riservazione oraria, Oraria &  \\ \hline
     Oggetto di & Associa una riservazione giornaliera a una sala & Riservazione giornaliera, Sala &  \\ \hline
     Oggetto di & Associa una fascia oraria a una sala & Oraria, sala &  \\ \hline
\end{longtable}

\newpage
\section{Progettazione ed integrazione delle viste}


\subsection{Integrazione delle viste} 
Descrizione e unificazione conflitti tra viste ....
\subsubsection{E/R completo}
\subsubsection{Dizionario entità}
\subsubsection{Dizionario relazioni}



%%%%%%%%%%%%%%%%%%%%%%%%%%%%%%%%%%%%%%%%%%%%%%%%%%%%%%%%%

\newpage
\addcontentsline{toc}{section}{Riferimenti bibliografici}
\bibliographystyle{apalike}
\bibliography{ref}
\begin{tabular}{c}
\end{tabular}
\\
Attributi delle entità: \href{https://ardisc.it/musitalia/vistaHomePage.php}{ArDisc} \\
Vari tipi di Album (Singolo, Extended Play e Long Play):  \href{https://blog.groover.co/it/consigli-per-i-musicisti/ep-vs-album-it/}{Groover} \\
Siti da cui abbiamo preso l'idea dei piani: \href{https://insoundstudio.com/studio/tariffe-e-offerte}{inSoundStudioProject}, \href{https://www.studioesagono.com/studio/rates/}{Studio Esagono} \\



\newpage
\section*{Contatti}

\begin{tabular}{l r l}
& Menozzi Matteo: & 316725@studenti.unimore.it \\
& Patrini Andrea: & 317106@studenti.unimore.it \\
& Turci Sologni Enrico: & 317448@studenti.unimore.it
\end{tabular}

\end{document}






%All other official TU Delft colours
\definecolor{donkerblauw}{RGB}{12, 35, 64}
\definecolor{turkoois}{RGB}{0, 184, 200}
\definecolor{blauw}{RGB}{0, 118, 194}
\definecolor{paars}{RGB}{111, 29, 119}
\definecolor{roze}{RGB}{239, 96, 163}
\definecolor{framboos}{RGB}{165, 0, 52}
\definecolor{rood}{RGB}{224, 60, 49}
\definecolor{oranje}{RGB}{237, 104, 66}
\definecolor{geel}{RGB}{255, 184, 28}
\definecolor{lichtgroen}{RGB}{108, 194, 74}
\definecolor{donkergroen}{RGB}{0, 155, 119}
%You can use these to change the hyperlink colour or the colour of the header or whatever. Glück Auf!
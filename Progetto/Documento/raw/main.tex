\documentclass{article}
\usepackage[italian]{babel}
\usepackage[utf8x]{inputenc}
\usepackage[T1]{fontenc}
\usepackage{graphicx}
\usepackage[colorinlistoftodos]{todonotes}
\usepackage[colorlinks=true, allcolors=tudelftblue]{hyperref} %sets hyperlink colour
\usepackage{caption}
\usepackage{subcaption}
\usepackage{xcolor}
\usepackage{roboto} % for Roboto Slab font
\usepackage{float}
\usepackage{titling} 
\usepackage{blindtext}\usepackage{titlesec}
\usepackage[square,sort,comma,numbers]{natbib}
\usepackage[colorinlistoftodos]{todonotes}
\usepackage{tikz}
\usepackage{geometry}
\usepackage{sectsty}
\usepackage{amsmath}
\usepackage{tikzpagenodes}
\usepackage{booktabs}
\usepackage{listings}
\definecolor{tudelftdarkblue}{RGB}{0,0,0}
\definecolor{tudelftcyan}{RGB}{209,65,36}
\definecolor{tudelftblue}{RGB}{99, 102, 106}
\geometry{a4paper, margin=2cm}
\allsectionsfont{\color{black}} %sets colour for all headers
\usepackage{helvet}
\renewcommand{\familydefault}{\sfdefault}
\sectionfont{\fontfamily{RobotoSlab-TLF}\selectfont}
%%%%%%%%%%%%%%%%%%%%%%%%%%%%%%%%%%%%%%%%%%%%%%%%%%%%%%%%%
\begin{document}

\input{titlepage}

%%% Create a table of contents
\tableofcontents
\newpage

\addcontentsline{toc}{section}{Introduzione}
\section*{Introduzione}






% \newpage
% \addcontentsline{toc}{section}{References}
% \bibliographystyle{apalike}
% \bibliography{ref}

\newpage
\section*{Contatti}

\begin{tabular}{l r l}
& Menozzi Matteo: & ??????@studenti.unimore.it \\
& Patrini Andrea: & ??????@studenti.unimore.it \\
& Turci Sologni Enrico: & 3174448@studenti.unimore.it
\end{tabular}


% \noindent Matteo Menozzi ??????@studenti.unimore.it \\

% \noindent Patrini Andrea ??????@studenti.unimore.it \\

% \noindent Turci Sologni Enrico 317448@studenti.unimore.it

\end{document}






%All other official TU Delft colours
\definecolor{donkerblauw}{RGB}{12, 35, 64}
\definecolor{turkoois}{RGB}{0, 184, 200}
\definecolor{blauw}{RGB}{0, 118, 194}
\definecolor{paars}{RGB}{111, 29, 119}
\definecolor{roze}{RGB}{239, 96, 163}
\definecolor{framboos}{RGB}{165, 0, 52}
\definecolor{rood}{RGB}{224, 60, 49}
\definecolor{oranje}{RGB}{237, 104, 66}
\definecolor{geel}{RGB}{255, 184, 28}
\definecolor{lichtgroen}{RGB}{108, 194, 74}
\definecolor{donkergroen}{RGB}{0, 155, 119}
%You can use these to change the hyperlink colour or the colour of the header or whatever. Glück Auf!